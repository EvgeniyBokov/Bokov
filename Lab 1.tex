\documentclass[11pt]{article}
\usepackage{amsmath,amssymb,amsthm}
\usepackage{algorithm}
\usepackage[noend]{algpseudocode} 

%---enable russian----

\usepackage[utf8]{inputenc}
\usepackage[russian]{babel}

% PROBABILITY SYMBOLS
\newcommand*\PROB\Pr 
\DeclareMathOperator*{\EXPECT}{\mathbb{E}}


% Sets, Rngs, ets 
\newcommand{\N}{{{\mathbb N}}}
\newcommand{\Z}{{{\mathbb Z}}}
\newcommand{\R}{{{\mathbb R}}}
\newcommand{\Zp}{\ints_p} % Integers modulo p
\newcommand{\Zq}{\ints_q} % Integers modulo q
\newcommand{\Zn}{\ints_N} % Integers modulo N

% Landau 
\newcommand{\bigO}{\mathcal{O}}
\newcommand*{\OLandau}{\bigO}
\newcommand*{\WLandau}{\Omega}
\newcommand*{\xOLandau}{\widetilde{\OLandau}}
\newcommand*{\xWLandau}{\widetilde{\WLandau}}
\newcommand*{\TLandau}{\Theta}
\newcommand*{\xTLandau}{\widetilde{\TLandau}}
\newcommand{\smallo}{o} %technically, an omicron
\newcommand{\softO}{\widetilde{\bigO}}
\newcommand{\wLandau}{\omega}
\newcommand{\negl}{\mathrm{negl}} 

% Misc
\newcommand{\eps}{\varepsilon}
\newcommand{\inprod}[1]{\left\langle #1 \right\rangle}

 
\newcommand{\handout}[5]{
  \noindent
  \begin{center}
  \framebox{
    \vbox{
      \hbox to 5.78in { {\bf Научно-исследовательская практика} \hfill #2 }
      \vspace{4mm}
      \hbox to 5.78in { {\Large \hfill #5  \hfill} }
      \vspace{2mm}
      \hbox to 5.78in { {\em #3 \hfill #4} }
    }
  }
  \end{center}
  \vspace*{4mm}
}

\newcommand{\lecture}[4]{\handout{#1}{#2}{#3}{Scribe: #4}{Название темы #1}}

\newtheorem*{theorem}{Теорема}
\newtheorem{lemma}{Лемма}
\newtheorem{definition}{Определение}
\newtheorem*{corollary}{Следствие}
\newtheorem{fact}{Факт}

% 1-inch margins
\topmargin 0pt
\advance \topmargin by -\headheight
\advance \topmargin by -\headsep
\textheight 8.9in
\oddsidemargin 0pt
\evensidemargin \oddsidemargin
\marginparwidth 0.5in
\textwidth 6.5in

\parindent 0in
\parskip 1.5ex

\begin{document}

\lecture{Теория делимости в целых числах}{Лето 2020}{}{Боков Евгений}

\begin{theorem}[2-9]
	Линейное Диофантово уравнение $ax + by = c$ тогда и только тогда имеет решение, когда $d | c$, где $d = \gcd(a, b)$. Если $x_0, y_0$ частные решения этого уравнения, тогда все остальные решения задаются как
	\[x = x_0 + (b|d)t, \qquad y = y_0 - (a|d)t\]
	для различных целых t.
\end{theorem}

\begin{proof}
	Чтобы подтвердить второе утверждение теоремы, предположим, что известно решение $x_0, y_0$ данного уравнения. Если $x', y'$ - другое решение данного уравнения, то тогда
	\[ax_0 + by_0 = c =ax' + by',\]
	что эквивалентно
	\[a(x' - x_0) = b(y_0 - y).\]
	Согласно Следствию Теоремы 2-4, существуют взаимно простые целые числа $r$ и $s$, что $a = dr, d = ds$. Подставив это в последнее записанное уравнение и сократив общий коэффициент $d$, поулчим
	\[r(x' - x_0) = s(y_0 - y')\]
	Теперь ситуация такова: $r | s(y_0 - y')$, с $\gcd(r, s) = 1$. Используя лемму Евклида, должно быть так, что $r | (y_0 - y')$; или, другими словами, $y_0 - y' = rt$ для некоторого целого t. Подставив, мы получаем
	\[x' - x_0 = st.\]
	Это приводит нас к формулам
	\[\begin{split}
		x' = x_0 + st = x_0 + (b|d)t,\\
		y' = y_0 - rt = y_0 - (a|d)t.
	\end{split}\]
	Легко заметить, что эти значения удовлетворяют Диофантову уравнению, независимо от выбора целого числа t; для,
	\[\begin{split}
	ax' + by' &  =  a[x_0 + (b|d)t] + b[y_0 - (a|d)t]\\
	& = (ax_0 + by_0) + (ab|d - ab|d)t\\
	& = c + 0 \cdot t = c.
	\end{split}\]
	Таким образом, существует бесконечное число решений данного уравнения, по одному для каждого значения t.
\end{proof}

\subparagraph{Пример 2-3}
Рассмотрим линейное диофантово уравнение
\[172x + 20y = 1000\]
Применяя алгоритм Евклида к оценке $\gcd(172, 20)$, мы находим, что
\[\begin{split}
	172 & = 8 \cdot 20 + 12,\\
	20 & = 1 \cdot 12 + 8,\\
	12 & = 1 \cdot 8 + 4,\\
	8 & = 2 \cdot 4,
\end{split}\]
откуда $\gcd(172, 20) = 4$. Так как $4 | 1000$, решение этого уравнения существует. Чтобы получить целое число 4 в виде линейной комбинации $172$ и $20$, мы работаем в обратном порядке, выполнив приведенные выше вычисления следующим образом:
\[\begin{split}
	4 & = 12 - 8\\
	& = 12 - (20 -12)\\
	& = 2 \cdot 12 - 20\\
	& = 2(172 - 8 \cdot 20) - 20\\
	& = 2 \cdot 172 + (-17)20.
\end{split}\]
Умножив это соотношение на $250$, мы получаем
\[\begin{split}
	1000  = 250 \cdot 4 & = 250[2 \cdot 172 + (-17)20]\\
	& = 500 \cdot 172 + (-4250)20,
\end{split}\]
так что $x = 500$ и $y = -4250$ обеспечивают одно решение рассматриваемого уравнения Диофанта. Все остальные решения выражаются
\[\begin{split}
	x & = 500 + (20 / 4)t = 500 + 5t,\\
	y & = -4250 - (172/4)t = -4250 - 43t
\end{split}\]
для некоторого целого числа $t$.

\qquad Немного дальнейших усилий приводит к получению решений в натуральных числах, если таковые существуют. Для этого t должно быть выбрано так, чтобы одновременно удовлетворять неравенствам
\[5t + 500 > 0, \quad -43t - 4250 > 0\]
или, что то же самое,
\[-98\frac{36}{43} > t > -100.\]
Поскольку $t$ должно быть целым числом, мы вынуждены сделать вывод, что $t = -99$. Таким образом, наше диофантово уравнение имеет единственное положительное решение $x = 5, y = 7$, соответствующее значению $t = -99$.

Может быть полезно записать форму, которую принимает Теорема 2-9, когда коэффициенты являются относительно простыми целыми числами.

\begin{corollary}
	Если $\gcd(a, b) = 1$ и если $x_0, y_0$ является частным решением линейного диофантова уравнения $ax + by = c$, то все решения определяются как
	\[x = x_0 + bt, \quad y = y_0 - at\]
	для целых начений $t$.
\end{corollary}

\qquad Например: уравнение $5x + 22y = 18$ имеет $x_0 = 8, y_0 = -1$ в качестве одного решения; из следствия полное решение задается как $x = 8 + 22t, y = -1-5t$ для произвольного $t$.

\qquad Диофантовы уравнения часто возникают при решении определенных типов традиционных «словесных задач», о чем свидетельствует наш следующий пример.

\subparagraph{Пример2-4}
Клиент купил дюжину кусочков фруктов, яблок и апельсинов за $\$1,32$. Если яблоко стоит на 3 цента дороже, чем апельсин и больше яблок, чем было куплено апельсинов, сколько штук каждого вида было куплено?

\qquad Чтобы установить эту проблему как диофантово уравнение, пусть $x$ будет количеством яблок, а $y$ - количеством купленных апельсинов; также, пусть $z$ представляет стоимость (в центах) апельсина. Тогда условия задачи приводят к
\[(z+3)x + zy = 132\]
или эквивалентно
\[3x + (x + y)z = 132\]
Поскольку $x + y = 12$, вышеприведенное уравнение можно заменить на
\[3x + 12z = 132,\]
что в свою очередь упрощает до $х + 4z = 44$.

Без лишних предметов цель состоит в том, чтобы найти целые числа $x$ и $z$, удовлетворяющие диофантову уравнению
\begin{equation*}
x+4z=44
\tag{*}
\label{stareq}
\end{equation*}

Поскольку $\gcd(1, 4) = 1$ является делителем $44$, у этого уравнения есть решение. Умножив соотношение $1 = 1 (-3) + 4 · 1$ на $44$, получим
\[44 = 1 (-132) + 4 \cdot 44\]
откуда следует, что xo = -132, zo = 44 служит одним из решений. Все остальные решения (\ref{stareq}) имеют вид
\[\begin{split}
	x & = -132 + 4t,\\
	z & = 44 - t,
\end{split}\]
где t является целым числом.

Не все из бесконечного множества значений дают решения исходной задачи. Следует учитывать только значения t, которые гарантируют, что $12 > = × > 6$. Это требует получения таких $t$, что
\[12 \geq -132 + 4t > 6.\]
Теперь $12> = - 132 + 4t$ означает, что $t \geq 36$, а $-132 + 4t> 6$ дает $t > 34 \frac{1}{2}$. Единственные целые значения $t$, чтобы удовлетворить оба неравенства, это $t = 35$ и $t = 36$. Таким образом, есть две возможные покупки: дюжина яблок по $11$ центов за штуку (случай, когда $t = 36$), или еще $8$ яблок по $12$ центов каждый и $4$ апельсина по $9$ центов каждый (случай, когда $t = 35$)
\begin{center}
	\item \paragraph{ЗАДАЧИ 2.4}
\end{center}

\begin{enumerate}
	\item Определите все решения в целых числах каждого из следующих диофантовых уравнений:
	\begin{enumerate}
		\item $56x + 72y = 40$;
		\item $24x + 138y = 18$;
		\item $221x + 91y = 117$;
		\item $84x - 438y = 156$.
	\end{enumerate}
	\item Определите все решения в натуральных числах каждого из следующих диофантовых уравнений:
	\begin{enumerate}
		\item $30x + 17y = 300$;
		\item $54x + 21y = 906$;
		\item $123x + 360y = 99$;
		\item $158x - 57y = 7$.
	\end{enumerate}
	\item Если $a$ и $b$ являются взаимно простыми положительными целыми числами, докажите, что в диофантовом уравнении $ax - by = c$ имеется бесконечно много решений в натуральных числах.
	
	[Подсказка: существуют целые числа $x_0$ и $y_0$, такие что $ax_0 + by_0 = 1$. Для любого целого числа $t$, которое больше обоих $| х_0 | / b$ и $| y_0 | / a$, $x = x_0 + bt$ и $y = - (y_0 - at)$ являются положительным решением данного уравнения.]
	\item
	\begin{enumerate}
		\item Докажите, что диофантово уравнение $ax + by + cz = d$ разрешимо в целых числах тогда и только тогда, когда $\gcd(a, b, c)$ делит $d$.
		\item Найдите все решения в целых числах $15x + 12y + 30z = 24$.
		
		[Подсказка: используйте $y = 3s - 5t$ и $z = -s + 2r$.]
	\end{enumerate}
	\item
	\begin{enumerate}
		\item Человек имеет $\$4,55$ в виде монет, состоящих исключительно из десяти центов и четвертаков. Какое максимальное и минимальное количество монет у него может быть? Может ли количество десятицентовиков быть равным количеству четвертаков?
		\item Театр в окрестностях платит $\$1,80$ за вход для взрослых и $75$ центов за детей. В определенный вечер общая сумма поступлений составила $\$90$. Предполагая, что присутствовало больше взрослых, чем детей, сколько человек присутствовало?
		\item Определенное количество шестерок и девяток добавлено, чтобы получить сумму 126; если число шестерок и девяток поменяно местами, новая сумма равна 114. Сколько из них было изначально?
	\end{enumerate}
	\item Фермер купил сто голов скота на общую сумму $\$4000$. Цены были следующими: телята по $\$120$ за каждого; ягнята, $\$50$ каждый; поросята по $\$25$ каждый. Если фермер получил хотя бы одно животное каждого типа, сколько он купил?
	\item Когда мистер Смит обналичивал чек в своем банке, кассир перепутал количество центов с количеством долларов и наоборот. Не подозревая об этом, мистер Смит потратил 68 центов, а затем, к своему удивлению, обнаружил, что у него вдвое больше первоначального чека. Определите наименьшее значение, на которое можно было бы выписать чек.
	
	[Подсказка: если $x$ - это количество долларов, а $y$ - количество центов в чеке, то $100y + x - 68 = 2 (100x+y)$.]
\end{enumerate}


\end{document}